\section{บทสรุป (Conclusion)}

รายงานฉบับนี้แสดงถึงบทบาทสำคัญของพีชคณิตเชิงเส้นต่อสถาปัตยกรรม Transformer ซึ่งเป็นรากฐานของโมเดลภาษาขนาดใหญ่ในงานประมวลผลภาษาธรรมชาติยุคปัจจุบัน

จากการศึกษา แนวคิดทางพีชคณิตเชิงเส้น เช่น เวกเตอร์ เมทริกซ์ ผลคูณจุด การแปลงเชิงเส้น และฟังก์ชัน Softmax เป็นกลไกหลักที่ช่วยให้ Transformer สร้างความหมายเชิงบริบทได้ โดยการรวมเชิงเส้นของเวกเตอร์ Value ตามน้ำหนักความสนใจ เรียนรู้ผ่าน Backpropagation ที่อาศัยกฎลูกโซ่และ Jacobian Matrices รวมถึงเข้าใจบริบทของภาษาโดยกลไก Self-Attention ซึ่งใช้การคูณเมทริกซ์ระหว่าง Query, Key และ Value เพื่อประเมินความสัมพันธ์ในประโยค พร้อมพิจารณาปัจจัยเพิ่มเติม เช่น ความใกล้ชิดเชิงตำแหน่ง บทบาททางไวยากรณ์ และความหมายเชิงบริบท

ทฤษฎีเหล่านี้เชื่อมโยงไปยังกิจกรรมเชิงประยุกต์ เช่น Chatbot อัจฉริยะ ระบบแปลภาษาอัตโนมัติ และการสรุปความ ซึ่งล้วนใช้พีชคณิตเชิงเส้นเพื่อสร้างสถาปัตยกรรม Transformer ที่จับความสัมพันธ์ระหว่างข้อมูล เลือกความหมายตามบริบท ประมวลผลแบบขนาน จัดการความสัมพันธ์ระยะไกล และสังเคราะห์ข้อมูลเชิงบริบทได้มีประสิทธิภาพ

การบูรณาการพีชคณิตเชิงเส้นกับเทคโนโลยี Transformer จึงสร้างระบบที่เข้าใจและสื่อสารด้วยภาษามนุษย์ได้อย่างชาญฉลาด ถือเป็นก้าวสำคัญของวิวัฒนาการปัญญาประดิษฐ์ เพราะ \emph{Attention is all you need}.
