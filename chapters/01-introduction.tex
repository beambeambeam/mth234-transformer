\chapter{บทนำ (Introduction)}

\indent แบบจำลองภาษาขนาดใหญ่ (Large Language Model: LLM) ถือเป็นหนึ่งในนวัตกรรมสำคัญของยุคปัญญาประดิษฐ์ ที่สามารถเข้าใจ สร้างสรรค์ และโต้ตอบกับภาษามนุษย์ได้อย่างเป็นธรรมชาติ กลไกสำคัญเบื้องหลัง LLM คือสถาปัตยกรรม Transformer ซึ่งอาศัยแนวคิดทาง พีชคณิตเชิงเส้น (Linear Algebra) เป็นรากฐานในการประมวลผลข้อมูลเชิงภาษาในรูปแบบเวกเตอร์และเมทริกซ์อย่างมีประสิทธิภาพ

    ในสาขา วิทยาศาสตร์ข้อมูล (Data Science) และ ปัญญาประดิษฐ์ (Artificial Intelligence) แนวคิดของ Linear Algebra มีบทบาทสำคัญในการจัดการและแปลงข้อมูลจากคำหรือประโยคให้กลายเป็นตัวแทนเชิงตัวเลข (Numerical Representation) เพื่อให้คอมพิวเตอร์สามารถ “เข้าใจ” ความหมายของภาษาได้ เช่น การคำนวณเวกเตอร์ใน Embedding Space, การคูณเมทริกซ์ในกระบวนการ Self-Attention, และการรวมเชิงเส้น (Linear Combination) ของข้อมูลจากหลายแหล่ง

    โมเดล Transformer ได้เปลี่ยนแปลงแนวทางการประมวลผลภาษาธรรมชาติ (Natural Language Processing: NLP) อย่างสิ้นเชิง โดยอาศัยโครงสร้าง Attention ที่สามารถให้โมเดล “โฟกัส” กับส่วนสำคัญของประโยคได้อย่างยืดหยุ่น ซึ่งกระบวนการนี้ล้วนขับเคลื่อนด้วยการดำเนินการเชิงเส้น เช่น การคูณเมทริกซ์ระหว่าง Query, Key และ Value ตลอดจนการแปลงเชิงเส้นในแต่ละชั้นของเครือข่ายประสาทเทียม (Neural Network Layers)

    รายงานฉบับนี้ มุ่งเน้นการแสดงให้เห็นถึงบทบาทของพีชคณิตเชิงเส้นที่เกี่ยวข้องในกระบวนการทำงานของ Transformer โดยเริ่มจากการทบทวนความรู้พื้นฐานที่เกี่ยวข้อง เช่น เวกเตอร์ เมทริกซ์ และการแปลงเชิงเส้น จากนั้นอธิบายโครงสร้างหลักของโมเดล Transformer รวมถึงกลไกการทำงานของ Self-Attention และ Feed Forward Network พร้อมกรณีศึกษาการนำ LLM ไปประยุกต์ใช้ในงานจริง เช่น การสรุปเนื้อหาอัตโนมัติ การตอบคำถาม และการสร้างข้อความเชิงสร้างสรรค์

เนื้อหาในรายงานจะแบ่งออกเป็น 4 ส่วนหลัก ได้แก่

\begin{itemize}
    \item ความรู้พื้นฐานทางคณิตศาสตร์ที่เกี่ยวข้อง - ทบทวนแนวคิด Linear Algebra ที่เป็นหัวใจของการคำนวณใน Transformer

    \item โครงสร้างและกระบวนการทำงานของ Transformer - อธิบายส่วนประกอบหลัก เช่น Attention, Encoder-Decoder, และการแปลงเชิงเส้นในแต่ละชั้น

    \item กรณีศึกษาและการประยุกต์ใช้ LLM - วิเคราะห์ตัวอย่างการใช้โมเดล Transformer ในงานด้านภาษา โดยประกอบไปด้วย Chatbot และ ระบบแปล

    \item การวิเคราะห์และอภิปรายผล - พิจารณาข้อดี ข้อจำกัด และแนวทางการพัฒนา LLM ด้วยแนวคิดเชิงพีชคณิตในอนาคต
\end{itemize}

สุดท้าย รายงานนี้จะสรุปถึงความสำคัญของการบูรณาการพีชคณิตเชิงเส้นและเทคโนโลยี Transformer ในการสร้างระบบที่สามารถเข้าใจและสื่อสารด้วยภาษามนุษย์ได้อย่างชาญฉลาด ซึ่งเป็นก้าวสำคัญของวิวัฒนาการปัญญาประดิษฐ์ในยุคปัจจุบัน.

